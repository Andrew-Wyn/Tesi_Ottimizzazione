\documentclass[a4paper, 12pt]{article}
\usepackage[osf]{libertinus} %font generale del documento
\pagestyle{plain} %nessxun heading o foot particolare
\usepackage[fontsize=13pt]{scrextend} %dimensione font 
\usepackage[a4paper,top=3cm,bottom=3cm,left=3cm,right=3cm]{geometry} %impaginazione e margini documento
\usepackage{graphicx, wrapfig} %gestione immagini e grafiche

\begin{document}

\begin{titlepage} %crea l'enviroment
\begin{figure}[t] %inserisce le figure
    \centering\includegraphics[width=0.25\textwidth]{logo_unipg}
\end{figure}
\vspace{20mm}

\begin{Large}
 \begin{center}
	\textbf{Dipartimento di Matematica e Informatica\\ Corso di Laurea Triennale in Informatica\\}
	\vspace{20mm}
    {\LARGE{TESI DI LAUREA}}\\
	\vspace{10mm}
	{\huge{\bf Metodi Di Ottimizzazione Non Lineare su Varietà ed Applicazioni}}\\
\end{center}
\end{Large}


\vspace{30mm}
%minipage divide la pagina in due sezioni settabili
\begin{minipage}[t]{0.47\textwidth}
	{\large{\bf Relatore:\\ Prof. Bruno Iannazzo}}
\end{minipage}
\hfill
\begin{minipage}[t]{0.47\textwidth}\raggedleft
	{\large{\bf Candidato: \\ Luca Moroni\\ }}
	\vspace{5mm}
	{\large{\bf Matricola: \\ 311279\\ }}
\end{minipage}

\vspace{25mm}

\hrulefill

\vspace{5mm}

\centering{\large{\bf Anno Accademico 2020/2021 }}

\end{titlepage}

\tableofcontents
\newpage

\section{Obiettivo}

\subsection{Esempio di figura}


\end{document}